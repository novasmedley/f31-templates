\documentclass[11pt]{article}
\usepackage[margin=0.5in]{geometry}

%dummy text
\usepackage{lipsum}

% section title spacing
\usepackage{titlesec}
\titlespacing\section{0pt}{0pt plus 2pt minus 2pt}{-3pt plus 2pt minus 2pt}
\titlespacing\subsection{0pt}{1pt plus 2pt minus 2pt}{-4pt plus 2pt minus 2pt}
\titlespacing\subsubsection{0pt}{0pt plus 2pt minus 2pt}{-3pt plus 2pt minus 2pt}

\setlength\parindent{0pt} % No paragraph indents
\setcounter{section}{1} % Set section counter
\renewcommand{\thesubsection}{\thesection.\alph{subsection}} % Set subsections to alpha listing
\pagenumbering{gobble} % turn off page numbe

%%%%%%% Times and others (must compile under XeLaTeX):
\usepackage{fontspec}
\defaultfontfeatures{Ligatures=TeX} % to allow en dashes, must be before \setmainfont{}
\setmainfont{Times New Roman}

%%%%%%% space between paragraphs
\setlength{\parskip}{6pt} 

%%%%%%% spacing betwen itmes in lists
\usepackage{enumitem}
\setlist{nosep}

%%%%%%%%%%%%%%%%%%%%%%%%%%%%%%%%%%%%%%%%%%%%%%%%%%%%%%%%

\begin{document}

\section*{Relevance to Public Health}

\textbf{Describe the relevance of this research to public health in, at most, three sentences. For example,
NIH applicants can describe how, in the short or long term, the research would contribute to
fundamental knowledge about the nature and behavior of living systems and/or the application
of that knowledge to enhance health, lengthen life, and reduce illness and disability. If the
application is funded, this public health relevance statement will be combined with the project
summary (above) and will become public information.}

\lipsum[111]

\end{document}